\section{Design Tasks of \qevis{}} \label{sec:qevisgoals}

The \qevis{} project follows the guidelines of the design study methodology~\cite{sedlmair2012design}. 
During the \textit{Winnow} stage, we employed the strategy of ``talk with many but stay with few''. 
To start, we conducted semi-structured interviews with many potential users of \qevis{} from various backgrounds and collected the questions they commonly encounter in  daily work.  
%The representative questions include: ``Where does the query execution time go?'', ``What is the bottleneck of query execution?'', ``Which components are abnormal?'', ``How to explain the abnormal components?''. 
Then, we selected four domain experts as long-term collaborators, including two researchers (P1, P2) from academia and two engineers (P3, P4) from industry. P1 works on big data and database systems 
%and conducts both research and teaching in a university, 
while P2 has rich research experience in query optimization. P3 is the leader of a team that focuses on cloud computing, while P4 is a senior engineer working on distributed query system maintenance. Note that P1 is also an author of this paper.
According to the feedback from potential users and domain experts, we formulate three major design requirements for \qevis{} as follows:

\stitle{R1. Understand query execution}
%According to the query execution process in \autoref{sec:qryprocessing}, 
Query execution can be understood at two scales: job scale and task scale. 
As an overview, the timing information of the jobs and their dependencies can provide a big picture of query execution. In this view, the analysts can know the general time usage of each high-level component and identify execution bottlenecks by analyzing the job dependencies.
To gain a more accurate and comprehensive understanding of the query execution process, a micro-level view is necessary. This involves visualizing the execution progress and data dependencies of the atomic tasks. 
Such detailed visualization enables analysts to identify the required optimizations and explore potential solutions for addressing the bottlenecks.

\stitle{R2. Identify the component \qm{worth paying attention to}} Identifying key components is crucial for analysts to effectively navigate among the multiple levels of granularity. To facilitate this process, it is necessary to provide sufficient information that enables analysts to pinpoint the essential jobs, machines, and tasks that should be explored.

\stitle{R3. \qm{Reason about} specific execution patterns}
After identifying the suspicious jobs or tasks, the next step is to find the reasons that cause their abnormal behavior.
To accomplish this, analysts should analyze the execution of jobs and tasks, along with relevant information such as machine profiling. Moreover, all of the information should be appropriately visualized and coordinated to enable effective explorations.


Based on the requirements above, we formulate the design tasks as:

\stitle{T1 Multi-grained visualization}
To help analysts understand the query execution process (R1) and identify the key components (R2), it is necessary to display the execution process and key attributes at different granularities with effective visual designs:
\squishlist
\item[\textbf{T1.1}] \stitle{Show the overview of query execution}
\qevis{} should effectively visualize how the query logical plan is executed as an overview of query execution.
Specifically, the timing information (e.g., start time, end time, and time usage) of each job should be shown intuitively for users to understand which components are time-consuming (\textbf{T1.1.1}).
Other attributes such as the task parallelism and time usage percentage of each sub-phase of a job should be provided when users inspect a specific job (\textbf{T1.1.2}).
Moreover, the dependencies of the jobs should be shown clearly for analysts to know the logic position of a specific job (\textbf{T1.1.3}).

%\item[\textbf{T1.2}] \stitle{Support fine-grained task-level analysis} 
%\qevis{} should effectively visualize the timing information of many atomic tasks (\textbf{T1.2.1}) and their dependencies (\textbf{T1.2.2}).
%Specifically, the system should enable users to understand the execution process and potential problems by presenting the clear visual patterns. 
\qm{
\item[\textbf{T1.2}] \stitle{Support fine-grained task-level analysis} 
\qevis{} should effectively visualize the timing information of many atomic tasks (\textbf{T1.2.1}) and their dependencies (\textbf{T1.2.2}). 
The different execution statuses should be revealed by distinctive visual patterns provided by the system. 
%Specifically, the system should enable users to identify the execution status by presenting the distinctive visual patterns. 
Moreover, the users should be able to identify the key tasks affecting the whole execution progress and drive the potential problems causing the performance issues. 
}
\squishend

\stitle{T2 Anomaly score and visualization}
To help users navigate among different analysis granularities and identify the components that are \qm{worth paying attention to} (R2), general anomaly scoring methods should be integrated to find the jobs and machines that have abnormal behavior and effectively narrow down the scope of analysis. 

\stitle{T3 Enable pattern reasoning} 
To reason about the execution patterns (R3), auxiliary information (e.g., such as data and system profiling) should be provided on demand, and this information should be properly linked with the execution visualizations for interactive exploration. 

\squishlist
\item[\textbf{T3.1}] \stitle{Provide rich auxiliary information to reason about execution anomalies}
\qevis{} should provide auxiliary information to help users understand query execution process and interpret anomalies. The distribution of task attributes, such as data input, data output and time usage, should be provided to reason about system behavior (\textbf{T3.1.1}). Moreover, machine profiling (e.g., memory and CPU usage) should be visualized jointly with the tasks such that users can correlate task execution with machine status (\textbf{T3.1.2}).
\item[\textbf{T3.2}] \stitle{Provide coordinated linkage among different views}
When inspecting a specific pattern or abnormal component, users need to switch among different views. 
Related components (e.g., job, task and machine) in different views should be well linked to make interactive exploration easy. 

\squishend